\chapter{Printer Calibration}\label{chapter:printer_calibration}

As you may know, printers do not print your PDF file exactly.
They will scale it to match their own preset configurations and potentially add padding spaces around the edges.
There is no way to control for this as every printer is unique and there are no base standards.
The only thing a user can do is have their generated PDF file have the correct distances and then ask that the person printing their document calibrate the printer accordingly.

\printercalibration{}

\vspace{2in}
This chapter will provide printer calibrations.
All the \textcolor{red}{red lines} drawn are $1$~inch in length.
All the \textcolor{blue}{blue lines} drawn are $2$~inches in length.
These are drawn from the edge of the document in TikZ, which has good distance metrics built into it, so these distances are accurate.
Print this page and measure the margins and the length of the arrows.
If your measurements do not match those printed \textbf{you need to calibrate your printer}.
It is probable that your printer has an option that is along the lines of ``actual size'', so that might be a good starting point.
It can also help to turn on \verb|\geometry{showframe=true}| in your preamble.
